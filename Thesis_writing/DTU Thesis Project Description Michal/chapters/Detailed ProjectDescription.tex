% Introduction of what we want to do
\chapter{Detailed Project Description}

While previous integrations of GGPO with Unity's object-oriented code have been successful, the compatibility of GGPO with Unity's Data-Oriented Technology Stack (DOTS) remains unexplored. The potential synergy between GGPO and DOTS presents a compelling opportunity to elevate the capabilities of both frameworks.

DOTS, still in development, provides a promising platform to investigate how it can complement the strengths of GGPO. Specifically, the focus lies on integrating GGPO's predictive input models within the framework of DOTS. With DOTS' enhanced performance capabilities, there exists a promising prospect to facilitate the development of multiplayer games featuring a multitude of entities with minimal latency.

Furthermore, the utilization of DOTS may provide additional resources for exploring player prediction techniques. However, existing research like \cite{Improving_prediction} suggests that while there may be improvements in prediction accuracy, the incremental gain compared to naive models may not be substantial, potentially necessitating additional computational resources. Additionally, there is a noticeable scarcity of scholarly literature and practical implementation examples concerning rollback netcode with prediction within contemporary game engines.

Hence, this thesis aims to bridge this gap by investigating the integration of GGPO with Unity's DOTS framework. Ultimately, the goal is to facilitate the creation of large-scale multiplayer games that leverage the strengths of both DOTS and GGPO, thereby enriching the multiplayer gaming experience. Because of this the title of the thesis shouldn't be understand as simply decreasing the lag in the game (combination of DOTS + GGPO won't lead to reduce latency much. Existing GGPO approaches already reduce latency as much as humanly possible. Reduced CPU cost because of DOTS can lead to reduced latency, but only if the CPU + socket poor performance is wasting many milliseconds.). Thus what DOTS + GGPO offers is reduced CPU consumption and higher game scope/scale for smooth multiplayer games and it's also potentially easier to work with.\newline

Previous research in this domain has primarily concentrated on conventional Object-Oriented Design (OOD) paradigms for game development and networking. Early implementations of GGPO in Unity were largely rooted in the OOD approach, thus paving the way for further exploration of its application within emerging technologies like DOTS (Data-Oriented Technology Stack) \cite{UnityGGPO}.

Several GitHub repositories and community projects have emerged, offering implementations of rollback netcode within popular game engines like Unity and Unreal Engine. These implementations vary in quality, documentation, and compatibility with newer versions of the game engines. However, the literature on rollback netcode integration within game engines is relatively sparse, with only a few studies exploring its implementation and optimization like \cite{ECS_unity_implementation}, \cite{Lockstep_Unity}.

% Explaining what is the objective of this thesis
\section{Research Focus}

\subsection{Problem Statement}

This thesis addresses the imperative for effective and scalable solutions to overcome network latency and synchronization challenges in online multiplayer games. Right now it's really challenging to create a performant large scale multiplayer online game and many developers could benefit from a potential already integrated solution.

\subsection{Objective}

This thesis aims to explore and develop novel methodologies by integrating GGPO (or its custom implementation) with DOTS, with the specific goal of enabling seamless large scale multiplayer gaming experiences within the DOTS framework. The aim is to address the issue of lag in multiplayer online games and their size (adjecent CPU performance), thereby unlocking the potential for high-performance, large-scale multiplayer games utilizing Unity's Data-Oriented Technology Stack.

\subsection{Hypothesis}

The hypothesis guiding this research is as follows:

\textbf{H}: \textit{"The integration of rollback netcode, represented by GGPO, with Unity's DOTS framework will enable the creation of large-scale multiplayer real-time strategy (RTS) games with improved scalability and performance."}\newline

If hypothesis \textbf{H} is accepted, the project will explore the following research question:

\textbf{RQ1}: \textit{"How does the integration of GGPO with Unity's DOTS framework enhance the scalability and performance of multiplayer RTS games?"}\newline

If hypothesis \textbf{H} is rejected, the thesis will investigate the factors contributing to its rejection and seek alternative avenues for achieving the desired outcomes in multiplayer RTS game development.

\subsection{Research Questions}

The research questions underpinning this thesis are:

\begin{itemize}
\item How can the integration of GGPO solution be effectively realized within Unity's Data-Oriented Technology Stack (DOTS)?
\item What are the performance implications of merging GGPO with DOTS, particularly concerning gameplay responsiveness and scalability?
\item In what ways does the integration of GGPO with DOTS differ from traditional object-oriented approaches in terms of performance and scalability?
\item What are the best practices and challenges associated with implementing rollback netcode solutions within modern game development frameworks?
\end{itemize}