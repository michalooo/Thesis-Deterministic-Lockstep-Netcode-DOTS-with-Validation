\chapter{Introduction} 
% This section should introduce reader to the problem which this thesis is aim to solve, how the solution will look like and if there are any knowledge prerequisites, 

% Figure out some options regarding the topic. 
    % Network Latency Mitigation through Data-Oriented Design in Fast-Paced Multiplayer Online Games --> We are not decreasing latency
    % RTS & Fighting game scalability and accessibility via a GGPO implementation in Unity's Data Oriented Tech Stack (DOTS) --> Maybe too direct?
    % Exploring the Synergy between DOTS and GGPO for Enhanced scalability and accessibility in RTS & Fighting game


%Introduce the problem
In the competitive area of multiplayer online games, game engine creators constantly strive to enable game creators and easy implementation of games which are having high performance while minimized latency. This balance is essential not only for enabling the creation of vast and immersive gaming experiences but also for ensuring that players enjoy smooth, responsive gameplay. Despite the ammount of techniques developed to address these issues, recent developments have presented an interesting prospect of creating comprehensive, out-of-the-box solution within one of the most popular game engines that would seamlessly integrates performance optimization with effective latency mitigation.

%Introduce the solution
This master thesis, titled \textit{Network Latency Mitigation through Data-Oriented Design in Fast-Paced Multiplayer Online Games}, seeks to tackle the mentioned problem from the perspective of game engine creators. By using the the power of Unity's Data-Oriented Technology Stack (DOTS) framework, recently released after years of development which is known for its ability to alleviate CPU strain and accommodate large numbers of entities in a game, in combination with a custom implementation of GGPO, a proven networking solution capable of mitigating latency issues through player prediction, this thesis aims to provide an out-of-the-box solution for game developers.

The primary objective of this research is to demonstrate the feasibility and efficacy of said integration within Unity's game engine environment. Through experimentation and analysis, this thesis will not only ascertain the compatibility and performance of these technologies in their current states but also offer insights into potential enhancements and optimizations. By explaining the implementation details and methodologies employed, this research aims to empower game engine creators with the knowledge and tools necessary to leverage these advancements in their own projects.



% QUESTIONS
    % 1) Are game creators or game engine creators a center of the problem. For who is the solution?