\chapter{Introduction} 
% This section should introduce reader to the problem which this thesis is aim to solve, how the solution will look like and if there are any knowledge prerequisites?.

% Figure out some options regarding the topic. 
    % Network Latency Mitigation through Data-Oriented Design in Fast-Paced Multiplayer Online Games --> We are not decreasing latency
    % RTS & Fighting game scalability and accessibility via a GGPO implementation in Unity's Data Oriented Tech Stack (DOTS) --> Maybe too direct?
    % Exploring the Synergy between DOTS and GGPO for Enhanced scalability and accessibility in RTS & Fighting game


%Introduce the problem
In the competitive area of multiplayer online games, optimizing performance and minimizing latency in an out-of-the-box solution is a constant concern of game engine creators enabling their users to easily implement complex  games which are having high performance while also minimized latency. The balance between those 2 is essential not only for enabling the creation of vast and immersive gaming experiences but also for ensuring that players enjoy smooth, responsive gameplay. Despite the number of techniques developed to address these issues, recent developments have presented an interesting prospect of exploring a new combination of complementary softwares within one of the most popular game engines that could seamlessly integrate performance optimization with effective latency mitigation.

%Introduce the solution
This master thesis, titled \textit{Network Latency Mitigation through Data-Oriented Design in Fast-Paced Multiplayer Online Games}, seeks to tackle the mentioned problem from the perspective of game engine creators. By using the the power of Unity's Data-Oriented Technology Stack (DOTS) framework, recently released after years of development which is known for its ability to alleviate CPU strain and accommodate large numbers of entities in a game, in combination with a custom implementation of GGPO, a proven networking solution capable of mitigating latency issues through player prediction, this thesis aims to provide an comprehensive solution for game developers that would increase scalability and accessibility in their games.

The primary objective of this research is to demonstrate the feasibility and efficacy of said integration within Unity's game engine environment. Through experimentation and analysis, this thesis will not only ascertain the compatibility and performance of these technologies in their current states while also offering insights into potential enhancements and optimizations. By explaining the implementation details and methodologies employed, this research aims to equip game engine creators with the knowledge and tools necessary to leverage these advancements in their own projects.


% QUESTIONS
    % 1) Are game creators or game engine creators a center of the problem. For who is the solution?
    % 1) Can we get the benefits of both?
    % 2) How do we implement it?
    % 3) What are the benefits of both?
    % 4) What are the drawbacks of both?
    % 5) What are the drawbacks of the combination?
    % 6) What are the benefits of the combination?